\documentclass{article}

% XeTeX - for Unicode and TTF
\usepackage{fontspec}
\setmainfont[Ligatures=TeX]{Arial}

% For loading images
\usepackage{graphicx}

% Page Left Margin
%\addtolength{\hoffset}{-2cm}
% Text Width (and implicitly, Right Margin)
%\addtolength{\textwidth}{+4cm}

% Clickable TOC and references
\usepackage{hyperref}
\hypersetup{
	colorlinks,
    citecolor=magenta,
    filecolor=brown,
    linkcolor=blue,
    urlcolor=brown
}

% Make \paragraph a 4-th section, although it will not appear in TOC
\usepackage{titlesec}
\setcounter{secnumdepth}{4}
\titleformat{\paragraph}{\normalfont\normalsize\bfseries}{\theparagraph}{1em}{}
\titlespacing*{\paragraph}{0pt}{3.25ex plus 1ex minus .2ex}{1.5ex plus .2ex}

% Count \subparagraph, but without outline numbering
% Used for references to sub-paragraphs: You have to count the subparagraph (thus, don't put an *), and add a label after it. Then you can reference that label, and it will produce the number of the subparagraph. When you want normal subparagraph without counter, add the *, but you won't be able to reference it
\usepackage{chngcntr}
\setcounter{secnumdepth}{5}
\counterwithout{subparagraph}{paragraph}

\renewcommand\thesubsection{\thesection.\arabic{subsection}}

\usepackage{longtable}

% For easy calculations
\usepackage{calc}

% For coloring text
\usepackage[usenames,dvipsnames,svgnames,table]{xcolor}

\title{iLoc8 High Level Design \\ \large version 0.5 draft}
\author{Sorel Mitra}
\date{\today}

% Custom bullets
\usepackage{bbding}
\usepackage{pifont}
\usepackage{enumitem}

% Macros
\def\tbd #1 {\textbf{\color{red} TBD:} #1}

\begin{document}

	\maketitle
	
	\section*{Referenced Documents}
	
	\subparagraph{} Avaya-iLoc8-Roadmap-SAAED-2016-11-08.pptx \label{doc:roadmap}
	\subparagraph{} iLoc8\_Package\_SRS\_vX10.doc \label{doc:srs}
	\subparagraph{} EPT - iLoc8 Design Architecture – v7.pptx \label{doc:design_original}
	\subparagraph{} EPT - Englebart- v1.pptx \label{doc:design_engelbart}
	
	\section*{History}
	\begin{longtable}[l]{ | l | l | p{0.7\textwidth} |}
		\hline
		\textbf{Date} & \textbf{Version} & \textbf{Changes} \\ \hline
		\endhead
		2016-10-20 & 0.1 & Created. Added General Architecture, Overview. \\ \hline
		2016-10-24 & 0.2 & Added Server Application. \\ \hline
		2016-11-11 & 0.3 & Re-started. Added use cases, Product description. \\ \hline
		2016-11-15 & 0.4 & Added Architecture. Enhanced use cases. Added All Components. Added sections for Security, HA, Scalability\\ \hline
		2016-10-16 & 0.5 & Added Session Sharing mechanisms. Enhanced HA section. \\ \hline
	\end{longtable}
	
	\newpage
	
	\tableofcontents
	
	\newpage
	
	\section{Introduction}

	A citizen, denoted here as the Caller, has a phone audio discussion with the Agent. The phone discussion takes place outside of iLoc8. After mutual agreement, discussion is upgraded to have a data and video channels. iLoc8 starts after this upgrade (see Figure \ref{fig:usecase_general}), and offers the following features:
	
	\begin{itemize}
		\item Data channel with:
		\begin{itemize}
			\item Automatic upload of Caller's location, language, and device battery percentage
			\item Manual upload of photos, videos, and Caller's address
			\item Negotiating and managing the Video channel
			\item Text Chat
		\end{itemize}
		\item Video channel for a video conversation
		\item Storage for:
		\begin{itemize}
				\item Files
				\item Events
		\end{itemize}
	\end{itemize}
	
	\begin{figure}[htbp]
		\hspace{2cm}
		\includegraphics[width=\textwidth - 2cm]{usecase_general}
		\caption{The iLoc8 Solution}
		\label{fig:usecase_general}
	\end{figure}
    
	This document provides a high level view of the iLoc8 Solution:
	\begin{itemize}
		\item Use Cases
		\item Technical Solution
		\item Phases
	\end{itemize}
	
	\textbf{Note on Phases}: In [\ref{doc:roadmap}], there are a few phases outlined. This document all phases, unless specifically stated otherwise. Whenever the high-level design is different between phases, this is clearly specified.

	\section{Use Cases}

	\subsection{Start iLoc8}
	
	The Agent initiates iLoc8 from his computer. The Caller starts iLoc8 Dashboard by clicking on the SMS he receives.
	
	\begin{figure}[htbp]
		\hspace{2cm}
		\includegraphics[width=\textwidth - 2cm]{usecase_start}
		\caption{Start iLoc8}
		\label{fig:usecase_start}
	\end{figure}

	\newpage
	
	\subsection{iLoc8 Features}
	
	Main features of iLoc8.
	
	\begin{figure}[htbp]
		\hspace{1.5cm}
		\includegraphics[width=\textwidth - 1.5cm]{usecase_features}
		\caption{Use iLoc8}
		\label{fig:usecase_features}
	\end{figure}

	\newpage
	
	\subsection{Agent Comes Back}
	
	Agent comes back after a failure of his dashboard.
	
	\begin{figure}[htbp]
		\hspace{2cm}
		\includegraphics[width=\textwidth - 2cm]{usecase_agent_back}
		\caption{Use iLoc8}
		\label{fig:usecase_agent_back}
	\end{figure}

	\newpage
	
	\subsection{Agent Transfer}
	
	Agent transfers iLoc8 session to another Agent. (Phone call transfer is not shown.)
	
	\begin{figure}[htbp]
		\hspace{2.5cm}
		\includegraphics[width=\textwidth - 2.5cm]{usecase_agent_transfer}
		\caption{Use iLoc8}
		\label{fig:usecase_agent_transfer}
	\end{figure}

	\newpage

	\section{Product}
	
	\subsection{Overview}
	
	The iLoc8 Product consists of three components: A Server and two Clients: Caller and Agent.

	The Server consists of:
	\begin{itemize}
		\item Physical Deployment
		\item Supporting Software
		\item Server Application
	\end{itemize}

	The Caller consists of an application that runs on the citizen's smart phone, and which connects to the Server. The application can be:
	\begin{itemize}
		\item JavaScript-based, running in a browser
		\item Native, for Android and iOS
	\end{itemize}
	
	The Agent consists of an application that runs on the agent's computer, and which connects to the Server. The application can be:
	\begin{itemize}
		\item JavaScript-based, running in the Chrome browser
		\item C\#-based, running in Interaction Center
	\end{itemize}

	\subsection{Server}
	
	\subsubsection{Physical Deployment}
	
	\subparagraph*{Phase 1}
	
	The Physical Deployment consists of:
	\begin{itemize}
		\item The iLoc8 Hardware: VMWare ESXi virtualization server
		\item Customer Hardware for File and Event Storage
		\item A network infrastructure provided and configured by the Customer, for connecting the iLoc8 and Customer Hardware
		\item Reverse Proxy Hardware
		\item Documentation on how to configure the iLoc8 and Reverse Proxy Hardware
	\end{itemize}

	\subparagraph*{Phase 2}
	
	The same as Phase 1

	\subparagraph*{Phase 3}
	
	\textbf{\color{red} TBD}: HA might require changes to Physical Deployment

	\subsubsection{Supporting Software}
	
	\subparagraph*{Phase 1}
	
	The Supporting Software is:
	\begin{itemize}
		\item A single Avaya Breeze node installed in the iLoc8 HW, for running the Server Components
		\item Avaya SBC, installed in the Reverse Proxy HW
		\item Customer's Web API and Software for File and Event Storage, installed in Customer's HW
		\item Documentation on how to configure the Avaya Supporting Software
	\end{itemize}
	
	\subparagraph*{Phase 2}
	
	The same as Phase 1

	\subparagraph*{Phase 3}
	
	\textbf{\color{red} TBD}: HA might require changes to Supporting Software

	\subsubsection{Server Application}

	\subparagraph*{Phase 1}
	
	The Server Application consists of:
	\begin{itemize}
		\item Server Components, running on the Supporting Software
		\item A Server API for the Caller
		\item A Server API for the Agent
		\item Documentation on how to:
		\begin{itemize}
			\item Install the Server Components
			\item Use the API
		\end{itemize}
	\end{itemize}

	\subparagraph*{Phase 2}
	
	The same as Phase 1

	\subparagraph*{Phase 3}
	
	The same as Phase 2

	\subsection{Caller}
	
	\subparagraph*{Phase 1}
	
	The Caller consists of:
	\begin{itemize}
		\item A Caller SDK that:
		\begin{itemize}
			\item Uses the Server API
			\item Allows for implementation of a Caller Application
		\end{itemize}
		\item The Caller Application, developed over the Caller SDK
	\end{itemize}

	\subparagraph*{Phase 2}
	
	The same as Phase 1

	\subparagraph*{Phase 3}
	
	The same as Phase 1

	\subsection{Agent}
	
	\subparagraph*{Phase 1}
	
	The Agent consists of:
	\begin{itemize}
		\item An Agent SDK that:
		\begin{itemize}
			\item Uses the Server API
			\item Allows for implementation of an Agent Application
		\end{itemize}
		\item The Agent Application, developed over the Agent SDK
	\end{itemize}

	\newpage

	\section{Architecture}
	
	\subsection{Bird's Eye View}
	
	An overview of the iLoc8 architecture in the general context is shown in Figure \ref{fig:architecture_birds_eye}.
	
	\begin{figure}[htbp]
		\hspace{-2cm}
		\includegraphics[width=\textwidth + 5cm]{architecture_birds_eye}
		\caption{Bird's Eye View}
		\label{fig:architecture_birds_eye}
	\end{figure}

	\newpage
	
	\subsection{System Components}
	
	The components of the iLoc8 solution are shown in Figure \ref{fig:architecture_components}. There are three types of components:
	\begin{itemize}
		\item Services: They do the main work
		\item Adaptors: The components that unify access to the various non-iLoc8 interfaces that the Core and Clients use
		\item Clients: The Caller and Agent applications
	\end{itemize}
	
	\begin{figure}[htbp]
		\hspace{-2cm}
		\includegraphics[width=\textwidth + 5cm]{architecture_components}
		\caption{Components}
		\label{fig:architecture_components}
	\end{figure}

	\subsection{Non-iLoc8 Service Adaptors}
	
	iLoc8 Adaptors hide details of accessing non-iLoc8 services. Each adaptor is designed for a particular service and provides:
	\begin{itemize}
		\item An uniform interface for accessing that service
		\item Implementation particular to that service
	\end{itemize}
	
	Each iLoc8 installation comes with its own version of Adaptors. Thus, the Adaptors to be used within the iLoc8 solution are chosen at install-time rather than at compile-time or runtime (Figure \ref{fig:architecture_adaptors}).

	\textbf{Note}: It is part of the Detailed Design to find a solution to assure uniformity of the Adaptor's interface and avoid duplication of interfaces.

	\begin{figure}[htbp]
		\hspace{0cm}
		\includegraphics[width=\textwidth + 1cm]{architecture_adaptors}
		\caption{Non-iLoc8 Service Adaptors}
		\label{fig:architecture_adaptors}
	\end{figure}
	
	\newpage
	
	\subsubsection{Media Store Adaptor}
	
	The Media Store Adaptor is a particular case of the Adaptors, in that, aside from the standard interface for Core, it also offers a File Transfer API for the Clients

	\begin{figure}[htbp]
		\hspace{-3cm}
		\includegraphics[width=\textwidth + 6cm]{architecture_adaptor_media}
		\caption{Media Store Adaptor}
		\label{fig:architecture_adaptor_media}
	\end{figure}
	
	\subsection{Server API}
	
	\subsubsection{API Parts}
	
	The Server API has two parts (See Figure \ref{fig:architecture_server_api}):
	\begin{itemize}
		\item The Caller API. This is the API that the Caller uses for all its interaction with iLoc8. It offers two types of features:
		\begin{itemize}
			\item General Caller Features
			\item File Upload Features
		\end{itemize}
		\item The Agent API. Used by the Agent for all its interaction with iLoc8. It offers two types of features:
		\begin{itemize}
			\item General Agent Features
			\item File Download Features
		\end{itemize}
	\end{itemize}
	
	\begin{figure}[htbp]
		\hspace{0cm}
		\includegraphics[width=\textwidth + 2cm]{architecture_server_api}
		\caption{Server API Parts}
		\label{fig:architecture_server_api}
	\end{figure}

	There's a third part of the Server API: The Caller Web Loader, whose' purpose is just to load the Caller Web App into the Caller's browser. \\
	
	Externally, each Client sees its own API and only that.\footnote{How do we enforce this?}

	Also, the Client does not know to which iLoc8 Component is talking. Internally, the APIs are exposed by:
	\begin{itemize}
		\item iLoc8 Core, for the General APIs
		\item Media Store Adaptor, for the File Transfer APIs
	\end{itemize}
	
	\subsubsection{API Target Clients}
	
	The Server API is designed for usage from two Client application types:
	\begin{itemize}
		\item JavaScript Web Apps:
		\begin{itemize}
			\item Mobile
			\item Desktop
		\end{itemize}
		\item Native Apps:
		\begin{itemize}
			\item Windows - .NET
			\item iOS
			\item Android
		\end{itemize}
	\end{itemize}
	
	\subsubsection{API Type}
	
	The Server API:
	\begin{itemize}
		\item Is RESTful
		\item Uses long-polls
		\item Could implement \href{http://timelessrepo.com/haters-gonna-hateoas}{HATEOAS}\footnote{This requires more investigation in the detailed design phase}
	\end{itemize}
	
	Since REST APIs are based on simple HTTP requests, they're can be accessed from any modern language/framework, including JavaScript, .NET, iOS Objective-C/Swift, and Android Java.
	
	For long-polling (or Comet programming), there are options for each target client platform. Some of them are:
	\begin{itemize}
		\item JavaScript: Native support via Async request and Callbacks
		\item .NET: \href{https://msdn.microsoft.com/en-us/library/system.net.httpwebrequest.getresponse(v=vs.110).aspx}{HttpWebRequest.GetResponse()}
		\item iOS:  \href{http://stackoverflow.com/questions/6300756/long-polling-in-objective-c}{NSURLConnection sendSynchronousRequest}
		\item Android: One could simply use a Service to launch a Sync HTTP request
	\end{itemize}
	
	For the CometD library:
	\begin{itemize}
		\item This may be used by the Agent only\footnote{Need more investigation during detailed design}
		\item In .NET, you have \href{https://github.com/Oyatel/CometD.NET}{CometD.NET}
	\end{itemize}
	
	\section{Server Application}
	
	The Server Application is the central part of the iLoc8 Solution.
	
	\subsection{Responsibilities}
	
	\subparagraph*{Phase 1}
	
	The Server Application is responsible for:
	\begin{itemize}
		\item Implementing the Data Channel
		\item Negotiating and facilitating File Transfers between the Customer's storage and Clients
		\item Negotiating the Video Channel between Clients
		\item Saving events to Customer's storage
	\end{itemize}

	\subparagraph*{Phase 2}
	
	The same as Phase 1, plus:
	\begin{itemize}
		\item Session Transfer from Agent to Agent
	\end{itemize}

	\subparagraph*{Phase 3}

	The same as Phase 2, plus:
	\begin{itemize}
		\item Facilitate Video Recording on the Customer's storage
		\item Session Transfer with Conference
		\item \tbd{Video + audio? Not clear to me}
	\end{itemize}

	\subsection{Components}
	
	The responsibilities of the Server Application are split between the Breeze components shown in Figure \ref{fig:architecture_components}.
	
	There are two types of server components:
	\begin{itemize}
		\item Services, that do the main work
		\item Adaptors, that provide an uniform interface to the non-iLoc8 services
	\end{itemize}
	
	\subsubsection{iLoc8 Core}
	
	\subparagraph*{Phase 1}
	
	The iLoc8 Core is a Service that performs the main activities of the Server Application. Its roles are:
	\begin{itemize}
		\item Send SMS to the Caller via the SMS Adaptor
		\item Create, maintain, and delete sessions between Caller and Agent
		\item Implement Data Channel with Caller and Agent, except for File Transfers
		\item Negotiate File Transfers with Caller and Agent
		\item Negotiate Video Channel between Caller and Agent
		\item Serve Web version of the Caller Application
		\item Supervise and approve uploads and downloads done to the Customer Storage via the Media Store Adaptor
		\item \tbd{Anything for the languages, besides the Caller Dashboard choosing language?}
		\item Save Events to Customer Storage
		\item Maintain Event Numbering for all events and communicate those event numbers to the Agent SDK
	\end{itemize}
	
	\subparagraph*{Phase 2}
	
	\tbd{}
	
	\subparagraph*{Phase 3}
	
	\tbd{}
	
	\paragraph{Data Channel}
	
	The Data Channel consists of:
	\begin{itemize}
		\item Automatic sending of data from Caller to Server:
		\begin{itemize}
			\item GPS Location
			\item Battery Status
			\item Preferred Language
		\end{itemize}
		\item Manual sending from Caller to Server:
		\begin{itemize}
			\item Address
			\item File Uploads
		\end{itemize}
		\item File Downloads by Agent
		\item Textual Chats between Caller and Agent
		\item Events:
		\begin{itemize}
			\item Events generated by Server for all the Caller's activity. They are sent to the Agent
			\item Events generated by Server for failures. They are sent to both Caller and Agent to allow them to retry
			\item All Events are saved by Server to Customer Storage via the Event Adaptor
		\end{itemize}
	\end{itemize}
	
	\paragraph{Video Channel}
	
	\tbd{}
	
	\paragraph{Event Numbering}
	
	The Core assigns numbers to all events that come from the Caller.
	These numbers are communicated to the Agent SDK, and are used to put an Agent up-to-date when it comes back after a failure.
	
	\subsubsection{Media Store Adaptor}
	
	The Media Store Adaptor is an Adaptor for the Customer's Media Storage. It provides:
	\begin{itemize}
		\item An uniform Interface for using the Media Storage
		\item Several Implementations for particular Media Storages used by various Customers
		\item A REST Web API for File Transfers
		\item Integration with iLoc8 Core to allow the latter to supervise and approve File Transfers
	\end{itemize}
	
	\subsubsection{Event Adaptor}
	
	The Event Adaptor is an Adaptor for the Customer's Event Storage. It provides:
	\begin{itemize}
		\item An uniform Interface for using the Event Storage
		\item Several Implementations for particular Event Storages used by various Customers
	\end{itemize}

	\subsubsection{SMS Adaptor}
	
	The Event Adaptor is an Adaptor for the Customer's SMS Service. It provides:
	\begin{itemize}
		\item An uniform Interface for using the SMS Service
		\item Several Implementations for particular SMS Services used by various Customers\footnote{We could just get away with using the Breeze SMS API}
	\end{itemize}

	\section{Caller Application}

	\subsection{Caller SDK}

	The Caller SDK is a set of interfaces and implementations, used for developing the Caller Application.
	
	\subparagraph*{Phase 1}
	
	The Caller SDK provides the following functionalities:
	\begin{itemize}
		\item Register with the iLoc8 Server
		\item Get location, battery, and language from the smartphone OS
		\item Automatically send location, battery, and language
		\item Automatically change UI language to match the OS when possible
		\begin{itemize}
			\item Notify upper layer when this is not possible, along with the list of available languages
			\item \tbd{Is this part of SDK, or App?}
		\end{itemize}
		\item Upload file from smartphone to a server
		\item Send an Address Location to the iLoc8 Server
		\item Notify the upper layer of the following internal events:
		\begin{itemize}
			\item Language mismatch \& list of available languages
			\item Automatic info sent (location, battery, language)
			\item Manual info sent (location)
			\item Upload sent
			\item Video Conference started/ended
		\end{itemize}
		\item Long-polling for subscribing to:
		\begin{itemize}
			\item Failure Events for its own activities (such as upload, text chat)
		\end{itemize}
		\item Process failure events sent by Server and:
		\begin{itemize}
			\item For failure of the automatic sending of data:
			\begin{itemize}
				\item Automatically retry sending it a number of times
				\item Notify the upper layer after that number of retries
				\item Allow for restarting automatic retrying
			\end{itemize}
			\item Notify the upper layer about failure of manually-initiated events
		\end{itemize}
	\end{itemize}

	The Caller SDK is developed in:
	\begin{itemize}
		\item JavaScript\footnote{Or enhanced related frameworks, such as AngularJS} for Mobile Web
		\item Objective-C for iOS
		\item Java for Android
	\end{itemize}
	
	\subparagraph*{Phase 2}
	
	\tbd{}

	\subparagraph*{Phase 3}

	\tbd{}
	
	\subsection{Caller Reference Application}
	
	The Caller Reference Application is a fully functional sample application, whose purpose is to show how to use the SDK, and demo the whole iLoc8 solution.
	
	\subsection{Caller Application}
	
	The caller application is developed in JavaScript\footnote{Or enhanced related frameworks, such as AngularJS}, using the Caller SDK.
	
	
	Each customer will have their own customised version of the Caller Application.
	
	\section{Agent Application}

	\subsection{Agent SDK}

	The Agent SDK is a set of interfaces and implementations, used for developing the Agent Application.
	
	\subparagraph*{Phase 1}
	
	The Agent SDK provides the following functionalities:
	\begin{itemize}
		\item Register with the iLoc8 Server
		\item Long-polling for subscribing to:
		\begin{itemize}
			\item Data Channel Events from the Server:
			\begin{itemize}
				\item Receive location, battery, preferred language
				\item Caller Dashboard Language Change
				\item Receive manually input location
				\item Pre-recorded video upload start/end/fail
				\item Picture upload start/end/fail
				\item Video Conference started/ended
				\item \tbd{Phone details?} - see §2.1.3 from [\ref{doc:srs}]
			\end{itemize}
			\item Failure Events for its own activities (such as download, text chat)
		\end{itemize}
		\item A notification mechanism for the upper layer, that includes event types and data
		\item Download file
		\item Event Numbering:
		\begin{itemize}
			\item Store the number for the latest event it has
			\item Request for event re-sending when the latest event number it has is lower than the latest event number received from the server
		\end{itemize}
	\end{itemize}

	The Agent SDK is developed in C\#, and offers Programming Interfaces in C\#.
	
	\subparagraph*{Phase 2}
	
	The same as Phase 1, plus:
	\begin{itemize}
		\item \tbd{Video negotiation}
		\item \tbd{Video streaming}
	\end{itemize}

	\subparagraph*{Phase 3}

	\tbd{}
	
	\subsection{Agent Reference Application}
	
	The Agent Reference Application is a fully functional sample application, whose purpose is to show how to use the SDK, and demo the whole iLoc8 solution.
	
	\subsection{Agent Application}
	
	The agent application is developed in C\#, using the Agent SDK.
	
	Each customer will have their own customised version of the Agent Application.
	
	\section{Scalability and Performance}
	
	\subparagraph*{Phase 1}

	There is a single Breeze node. The system will scale as much as this single node allows it. The performance will be limited by the same.
	
	\subparagraph*{Phase 2}

	The same as Phase 1.
	
	\subparagraph*{Phase 3}

	The product runs in a Breeze Cluster of multiple nodes. Because of this, scalability and performance is be assured by:
	\begin{itemize}
		\item The Cluster's ability to add more nodes
		\item The Cluster's Load Balancer that will distribute load to all nodes
		\item The iLoc8's mechanism of \nameref{sec:session_sharing} that allows any node to access any session
	\end{itemize}
	
	Although it is acknowledged that using another load balancer would mean loosing the Breeze Cluster, an alternative solution is shown below, for completenes.
	
	\subparagraph*{Alternative Load Balancer \& Reverse Proxy}
	
	\href{http://www.haproxy.org}{HAProxy} is a:
	\begin{itemize}
		\item Load balancer with 3 modes of operation: round-robin, station, less loaded
		\item TLS/SSL terminator: install certificates only on HAProxy
		\item HTTP header manipulation
		\item Reverse Proxy
	\end{itemize}
	
	\tbd{Feasibility of having two load-balancers, or of disabling HAProxy's load balancer}
	
	\section{High-Availability and Failover}
	
	\subparagraph*{Phase 1}

	There is a single Breeze node. Because of this, there will be no HA in this phase, nor failover.
	
	\subparagraph*{Phase 2}

	The same as Phase 1.
	
	\subparagraph*{Phase 3}

	A \nameref{sec:session_sharing} mechanism is added that allows any node to access any session. Because of this, the other nodes can take over when a node fails.
	
	\section{Session Sharing}
	\label{sec:session_sharing}
	
	\subsection{Session Sharing Options}
	
	The following options for session sharing were investigated:
	\begin{itemize}
		\item Apache Ignite
		\item Gigaspaces XAP
		\item Redis
		\item Avaya Context Store
	\end{itemize}
	
	\subsubsection{Key Features}
	\label{sec:key_features}
	
	The mechanisms above are analysed in the following sections for the following key features:
	\begin{itemize}
		\item Speed: Access speed to data
		\item Persistence: Ability to persist data
		\item HA: Ability to get data when node(s) fail
		\item Usability: How we use this for iLoc8
		\item Integration: With Breeze
		\item Complexity: How complex is this solution
	\end{itemize}
	
	\subsubsection{Apache Ignite}
	
	\subparagraph*{Description}
	
	\href{https://ignite.apache.org}{Apache Ignite} is an in-memory \href{https://en.wikipedia.org/wiki/Data_grid}{data-grid} framework. It also has tailored usage for various scenarios, including \href{https://ignite.apache.org/use-cases/caching/web-session-clustering.html}{Web Session Clustering}.
	
	\subparagraph*{Speed}
	
	Ignite claims that access to the distributed data stored in it is fast.
	
	\subparagraph*{Persistence}
	
	Ignite integrates very well with \href{https://ignite.apache.org/features/igfs.html}{Apache Hadoop}.
	
	\subparagraph*{HA}
	
	Ignite assures HA by means of being distributed and replicating data.
	
	\subparagraph*{Usability}
	
	For iLoc8 session sharing, Ignite would be used as a distributed session storage mechanism with replication, where sessions would be stored in-memory on Ignite instances on each Breeze node (and possibly other nodes).
	
	\subparagraph*{Integration}
	
	Inside Breeze, Ignite would be used:
	\begin{itemize}
		\item Either from Maven or as a Standalone JAR
		\item Either as a separated service running from Linux
	\end{itemize}
	
	\subparagraph*{Complexity}
	
	Ignite looks moderately complex - i.e. it is not more complex than the things it does require it to be. Various usage examples on their site look pretty simple.
	
	\subsubsection{Gigaspaces XAP}
	
	\subparagraph*{Description}
	
	\href{http://www.gigaspaces.com/xap-in-memory-caching-scaling/datagrid}{Gigaspaces XAP} is an in-memory data-grid framework. It also has a tailored usage for \href{http://docs.gigaspaces.com/xap110/global-http-session-sharing-overview.html}{session sharing}.
	
	\subparagraph*{Speed}
	
	Gigaspaces XAP claims that access to the distributed data stored in it is fast.
	
	\subparagraph*{Persistence}
	
	Gigaspaces can write-behind to a database for persistent storage.
	
	\subparagraph*{HA}
	
	Gigaspaces assures HA by means of being distributed and replicating data.
	
	\subparagraph*{Usability}
	
	For iLoc8 session sharing, Gigaspaces would be used as a distributed session storage mechanism with replication, where sessions would be stored in-memory on each Breeze node.
	
	\subparagraph*{Integration}
	
	Inside Breeze, Gigaspaces is already integrated.
	
	\subparagraph*{Complexity}
	
	Gigaspaces XAP looks complex. This may be due to it addressing a lot of things\footnote{E.g. Ignite delegates persistence to Hadoop, while Gigaspaces does persistence, too}, but this complexity might challenge the developers. 
	
	\subsubsection{Redis}
	
	\subparagraph*{Description}
	
	\href{http://redis.io}{Redis} is an in-memory data structure store. It has automatic session storing for Apache Tomcat, and can be used manually to store sessions otherwise.
	
	\subparagraph*{Speed}
	
	Being in-memory, Redis is fast.
	
	\subparagraph*{Persistence}
	
	Redis has a module for \href{http://redis.io/topics/persistence}{persistent storage}.
	
	\subparagraph*{HA}
	
	Redis was designed mostly to be locally used, but has some modules that extend it beyond local usage: \href{http://redis.io/topics/sentinel}{Sentinel} for high availability/monitoring, and \href{http://redis.io/topics/cluster-tutorial}{Cluster} for distributed storage.

	Despite all these modules, our impression is that Redis is mostly designed for local usage, and that using it in a distributed manner requires more work and is prone to bugs.
	
	\subparagraph*{Usability}
	
	For iLoc8 session sharing, it would be used to store sessions in a distributed, replicated manner on some/all nodes that run Breeze.
	
	\subparagraph*{Integration}
	
	Inside Breeze, Redis would be used as a separated service running from Linux.
	
	\subparagraph*{Complexity}
	
	Redis looks the most simple of all, but the HA modules might increase its complexity in unwanted ways.
	
	\subsubsection{Avaya Context Store}
	
	\subparagraph*{Description}
	
	\href{https://www.devconnectprogram.com/site/global/products_resources/avaya_breeze/avaya_snap_ins/context_store/overview/index.gsp}{Avaya Context Store} is a Breeze ``snap-in that enables context-sensitive, real-time customer contact information to be updated from multiple sources and shared between the various components and touch points in the enterprise through which a customer passes.''
	
	As opposed to Ignite and Gigaspaces XAP, Context Store is not a distributed in-memory storage mechanism, but is a standalone persistent caching mechanism, with high-availability and scalability in mind.
	
	\subparagraph*{Speed}
	
	Thus, it's slow when compared to in-memory mechanisms.
	
	Assuring performance when using it requires a session affinity load balancing mechanism, so that all requests for a session go to the same node. Thus, we only use Context Store as a backup in case of failure, and don't depend on it for session retrieval.
	
	\subparagraph*{Persistence}
	
	Built-in.
	
	\subparagraph*{HA}
	
	Built-in.
	
	\subparagraph*{Usability}
	
	For iLoc8 session sharing, it is not suitable except for a backup mechanism.
	
	\subparagraph*{Integration}
	
	Inside Breeze, Context Store would be used as a separated service running in a different VM, with a REST/SOAP interface.
	
	\subparagraph*{Complexity}

	Context Store looks complex to us.
	
	\subsection{Comparison of Session Sharing Options}
	
	The \nameref{sec:key_features} of the session sharing mechanisms analysed above are put in this table as PROS/CONS/Neutral at a glance.
	
	\begin{longtable}[l]{ | p{0.2\textwidth} | p{0.25\textwidth} | p{0.25\textwidth} | p{0.25\textwidth} |}
		\hline
		\textbf{Mechanism} & \textbf{PROS} & \textbf{CONS} & \textbf{Neutral}\\ \hline
		\endhead
		
		Apache Ignite
		&
		\begin{itemize}[label=\Checkmark]
			\item Speed
			\item Persistence
			\item HA
			\item Complexity
		\end{itemize}
		&
		&
		\begin{itemize}[label=\SquareShadowBottomRight]
			\item Usability
			\item Integration
		\end{itemize}
		\\ \hline
		
		Gigaspaces XAP
		&
		\begin{itemize}[label=\Checkmark]
			\item Speed
			\item Persistence
			\item HA
			\item Integration
		\end{itemize}
		&
		\begin{itemize}[label=\XSolidBrush]
			\item Complexity
		\end{itemize}
		&
		\begin{itemize}[label=\SquareShadowBottomRight]
			\item Usability
		\end{itemize}
		\\ \hline
		
		Redis
		&
		\begin{itemize}[label=\Checkmark]
			\item Speed
			\item Complexity
		\end{itemize}
		&
		\begin{itemize}[label=\XSolidBrush]
			\item Persistence
			\item HA
		\end{itemize}
		&
		\begin{itemize}[label=\SquareShadowBottomRight]
			\item Usability
			\item Integration
		\end{itemize}
		\\ \hline
		
		Avaya Context Store
		&
		\begin{itemize}[label=\Checkmark]
			\item HA
			\item Persistence
		\end{itemize}
		&
		\begin{itemize}[label=\XSolidBrush]
			\item Speed
			\item Usability
			\item Complexity
		\end{itemize}
		&
		\begin{itemize}[label=\SquareShadowBottomRight]
			\item Integration
		\end{itemize}
		\\ \hline
	\end{longtable}

	\section{Security}
	
	The security of this application is assured by:
	\begin{itemize}
		\item Using a Reverse Proxy
		\item Using HTTPS
		\item Following OWASP recommendations
	\end{itemize}
	
\end{document}
